\acresetall
%%=========================================
\chapter{Conclusions}\label{ch:Conclusion}

This thesis has investigated the effect that cultural and biological evolution has on language evolution, as well as how the relation between language evolution and social networks function. The first research question was answered thoroughly through the discussion of four state-of-the-art computational language evolution models and comparing them to each other \citet{lipowska2011naming, lekvam2014co, gong2011simulating, munroe2002learning}. Research question two was answered by presenting a computational model that was an extension of the work done by \citeauthor{lekvam2014co}. The naming game, social network, and evolutionary algorithm that \citeauthor{lekvam2014co} used, were also used by this computational model. However, the fitness function, the method for calculating the weight of a word, and the genome were changed in an attempt to make the model more realistic and to study the impact cultural and biological evolution has on language evolution. Research question three was concerned with how social networks have an impact on language evolution. This thesis' contribution to that field was the new fitness function which was designed to allow both extrovert and introvert strategies to be viable. This did provide different results than Lekvam had, but all experiments performed in this study evolved similar social networks. In order to conclude on how social networks and language impact each others evolution, more specific research on this topic needs to be performed.
 
The fitness function implemented seemed to fulfil its purpose, both extrovert and introvert strategies had advantages in this fitness function. The simulations preferred an extrovert strategy during the initial generations, but in the end, an introvert strategy was favoured. In real life, the vast majority of conversations a person conducts are with other persons that the person already knows, meaning that a introvert strategy is favoured.
 
Parent-child dialogues were the most important factor in order for the simulations to reach a consensus. This indicates that the initial dialogues a newborn conducts are of crucial importance for its language, and it was virtually impossible to learn a new language properly after having lived for two generations in the simulations. Two generations in the simulation is not equal to two generations in real life, it merely represents that it it is almost impossible to become fluent at a new language after having lived for a while. This result indicates that being taught the same languages during the early years is an important factor on the \textit{ontogenetic} time scale. Both in this model and in reality, language is easier to learn while one is young and the older one becomes, the harder it is to learn new languages.
 
The more the selection pressure was reduced, the more the simulations struggled to reach consensus. When the selection pressure was reduced, the effects of the \textit{phylogenetic} (biological evolution) and \textit{glossogenetic} (cultural evolution) time scales were reduced. This indicates that these two evolutionary forces have an impact on language evolution. It also indicates that the Baldwin effect \citep{baldwin1896new} affects the cultural and biological forces of language evolution. Other simulations have also concluded that the Baldwin effect has an impact on language evolution \citep{lipowska2011naming, zollman2010plasticity, chater2010language}.

The results from this model indicate that it partly simulates language evolution successfully. The model indicates that the first years are of great importance for an individual's language, that mostly acting introvertly is beneficial, and that the Baldwin effect has an impact on language evolution. However, some parts of the model were made too simplistic to simulate language evolution in a sufficiently realistic manner. These weaknesses will have to be improved upon in order to learn more about how human language has evolved into what it is today.
 
\section{Future Work}
This computational model also attempted to investigate the relation between social networks and language evolution. The social network evolved into a network where all agents were transitively connected in all simulations. The simulations represented a small population, theoretically living with no distance between each other. A small town would likely have a network where almost everyone was transitively connected to each other, but small subgroups might exist within the population. Extending the computational model with geographical distances within which groups would form and simulating movement of the groups would be interesting. Implementing geographical distance could provide a better understanding of the relation between social networks and language evolution.
 
The genome evolved towards one local maximum, and all agents practically ended with the same personality. That personality favoured an introvert strategy, speaking to its newborns, and a relatively high learning ability. The personality based genome worked well in terms of depicting the important values to have in this model, but it did not depict the large variance between personalities that exists in real life. In reality, a population has some traits that always are favourable, and some traits that have positive and negative aspects to them. Considering that evolutionary algorithms are intended to search for optimal solutions in a complex search space, it is natural that the variance in personality was not represented properly in this model. The model would have to be altered in a manner which allows more variance and multiple personality strategies to be viable in order to incorporate personalities into the model.  
 
The naming game used in this model was a substantial simplification of human language. There are several possible ways to improve the language used in this computational model. The simplest method would be to have more objects to name. In reality, new objects appear forcing the population to agree upon words for the new objects. To simulate this, the naming game could be extended by adding objects to name at certain points during a simulation. Replacing the naming game entirely could also be considered. Two other methods for simulating language was presented in the literature study: the word order regularity model made by \citet{gong2004computational} and the model using artificial neural networks as the agents’ language interpreters made by \citet{munroe2002learning}.


\begin{comment}
    \section{Research Questions}
    The research questions of this thesis are:
    
    \begin{centering}
        \begin{enumerate}
            %\item Investigate the state-of-the-art in computational models that simulate language evolution and discuss their strengths and weaknesses.
            %\item Investigate the state-of-the-art within the field of simulating language using computational models. The computational models' strengths and weaknesses are discussed.
            \item How are the state-of-art computational models in the field of simulating language designed and what are their strengths and weaknesses?
            %\item The computational model presented in \citet{lekvam2014co} will be extended in an attempt to learn more about how the evolutionary forces affect language evolution.
            \item How can one state-of-the-art computational model be extended in order to learn more about how the evolutionary forces affect language evolution?
            %\item The computational model will continue the trend of simulating the co-evolution of language and social networks.
            \item How does social networks have an impact on language evolution? 
        \end{enumerate}
    \end{centering}
\end{comment}