\acresetall
%%=========================================
\chapter{Conclusions}
This thesis has investigated the effect that cultural and biological evolution has on language evolution, as well as how the relation between language evolution and social networks function. The computational model presented in this thesis is an extension of the computational model made by \citet{lekvam2014co}. The naming game, social network, and evolutionary algorithm that \citeauthor{lekvam2014co} used, was used by this computational model. However, the fitness function, the method for calculating the weight of a word, and the genome was changed in an attempt to make the model more realistic and study the effects of cultural and biological evolution.
 
The fitness function implemented seemed to fulfil its purpose, both extrovert and introvert strategies had advantages in this fitness function. The simulations preferred an extrovert strategy during the initial generations, but in the end, an introvert strategy was favoured. In real life, the vast majority of conversations a person conducts are with others that person already knows, meaning that a introvert strategy is favoured.
 
Parent-child dialogues were the most important factor in order for the simulations to reach a consensus. This indicates that the initial dialogues a newborn conducts were of crucial importance for its language, and it was virtually impossible to learn a new language properly after having lived for two generations. This is similar in reality, where language is easier to learn while one is young and the older one becomes, the harder it is to learn new languages.
 
The more the selection pressure was reduced, the more the simulations struggled to reach a consensus. When the selection pressure was reduced, the effects of the \textit{phylogenetic} (Biological evolution) and \textit{glossogenetic} (Cultural evolution) time scales were reduced. This indicates that these two evolutionary forces have an impact on language evolution. It also indicates that the Baldwin effect \citep{baldwin1896new} affects the cultural and biological forces of language evolution. Other simulations have also concluded that the Baldwin effect has an effect on language evolution \citep{lipowska2011naming, zollman2010plasticity, chater2010language}.
 

 
\section{Future Work}
This computational model also attempted to investigate the relation between social networks and language evolution. The social network evolved into a network where all agents were transitively connected in all simulations. The simulations represented a small population, theoretically living with no distance between each other. A small town would likely have a network where almost everyone were transitively connected to each other, but small subgroups might exist within the population. Extending the computational model with geographical distances which would form groups and simulating movement of the groups would be interesting. Implementing geographical distance could provide a better understanding of the relation between social networks and language evolution.
 
The genome evolved towards one local maximum, and all agents practically ended with the same personality. That personality favoured an introvert strategy, speaking to its newborns, and a relatively high learning ability. The personality based genome worked well in terms of depicting the important values to have in this model, but it did not depict the large variance between personalities that exists in real life.  In reality, a population has some traits that always are favourable, and some traits that have positive and negative aspects to them. Considering that evolutionary algorithms are intended to search for optimal solutions in a complex search space, it is natural that the variance in personality was not represented properly in this model. The model would have to be altered in a manner which allows more variance and multiple personality strategies to be viable in order to incorporate personalities into the model.  
 
The naming game used in this model was a substantial simplification of human language. There are several possible ways to improve the language used in this computational model. The simplest method would be to have more objects to name. The naming game could also be extended by adding objects to name at certain points during a simulation to simulate  that, in reality new objects which need to be explained will appear over time. Replacing the naming game entirely could also be considered. Two other methods for simulating language was presented in the literature study; The word order regularity model made by \citet{gong2004computational} and the model using artificial neural networks as the agents’ language interpreters made by \citet{munroe2002learning}.
