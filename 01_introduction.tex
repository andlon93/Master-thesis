\acresetall
%%=========================================
\chapter{Introduction}
% The first chapter of a well-structured thesis is always an introduction, setting the scene with background, problem description, objectives, limitations, and then looking ahead to summarize what is in the rest of the report. This is the part that readers look at first—so make sure it hooks them!
The complexity of human language is one of the biggest distinctions between us and other animals \citep{hauser2002faculty}. Many have tried to explain how human language evolved into what it is today, but no one has managed to do so yet \citep{chomsky1986knowledge, pinker1990natural, muller1861theoretical}. In recent years, computational models have contributed to the field of language evolution. The main contribution has come from the computational models used to test the validity of language evolution theories through simulations.

This thesis will present four computational models that simulate language evolution. In order to understand the field of language evolution and the computational models, the necessary theory will be presented. This includes theories on language evolution, graph theory, social networks, and \acp{ga}. A theory concerning how to model personalities will also be presented.

A computational model based on the work by \citeauthor{lekvam2014co} will be presented. The model uses Lekvam's work through his model's evolutionary algorithm and naming game. This thesis will attempt to improve upon his computational model by adding a new \textit{fitness function}, a new method for calculating the weights in the social network, slightly alter how agents decide to have a conversation, and a new \textit{genome} will be presented. The new genome is based on \citet{schwartz1992unniversals} who formulated a theory which attempts to define personalities through $10$ values. His theory will be presented in Chapter~\ref{ch:background}.

%In order to understand the field of language evolution and how computational models simulating language evolution are designed, the necessary theory will be presented. This includes theories on language evolution, graph theory, social networks, and \acp{ga}. A theory concerning how to model personalities will also be presented.

%The computational model designed for this thesis uses Lekvam's work through his model's evolutionary algorithm and naming game. This thesis will attempt to improve upon his work by adding a new \textit{fitness function}, a new method for calculating the weights in the social network, slightly alter how agents decide to conduct a conversation, and a new \textit{genome} will be presented. The new genome is based on \citet{schwartz1992unniversals} who formulated a theory which attempts to define personalities through $10$ values. His theory will be presented in \label{ch:background}.


\section{Objectives}
The objectives of this thesis are:

\begin{centering}
    \begin{enumerate}
        \item Investigate the state of the art in computational models that simulate language evolution and discuss their strengths and weaknesses.
        \item The compuational model presented in \citet{lekvam2014co} will be extended in an attempt to learn more about how the evolutionary forces affect language evolution.
        \item The computational model will continue the trend of simulating the co-evolution of language and social networks.  
    \end{enumerate}
\end{centering}

\section{Contribution}
This thesis will contribute to the field of computational language evolution models from a computer scientists perspective. This thesis will contribute to the field by presenting some new ideas that might enhance our understanding of the effect the evolutionary forces has on language evolution and the co-evolution of language and social networks.

\section{Outline}
Chapter~\ref{ch:background} presents theories that are relevant to the computational model designed for this thesis. In Chapter~\ref{ch:Chapter3}, four recently designed computational models will be discussed. The design of the computational model used in this thesis is presented in Chapter~\ref{ch:Chapter4}. Chapter~\ref{ch:Chapter5} presents the results of the seven experiments that were performed using the model presented in the previous chapter. Then, in Chapter~\ref{ch:Chapter6} the results are discussed. Finally, in Chapter~\ref{ch:Chapter7} a summary and conclusion of the discussed results and recommendations for future work are presented.      

%\section{Background}
% In this section, you should present the problem that you are going to investigate or analyze; why this problem is of interest; what has, so far, been done to solve the problem, and which parts of the problem that remain.

%\section{Problem Formulation}
% You should define your problem in a clear an unambiguous way and explain why this is a problem, why it is of interest—and to whom. It is also important to delimit the problem area.

%\subsection{Literature Survey}
% You should here present the main books and articles that treat problems that are similar to what you are studying. If you, later in your thesis, describe the “state of the art” – with a detailed literature survey, you may just give a very brief survey here (approx. a quarter of a page). If this is the only literature survey, you need to go into more details. An objective of the literature survey is to show the reader that you are familiar with the main literature within your field of research – so that you do not “reinvent the wheel.”

% References to literature can be given in two different ways: 
% • As an explicit reference: It is shown by Lundteigen and Rausand (2008) and partly also by Rausand and Høyland (2004) that ...
% • As an implicit reference: It is shown (e.g., see Rausand and Høyland, 2004, Chap. 4) that ...

% In the example above, we have used “author-year” references, which is the preferred format.

% Remark: Following agreement with your supervisor, you may also refer by numbers, for example, [1]. 

% When you refer to the scientific literature, you should always write in present tense. Example: Rausand and Høyland (2004) show that ...

%\subsection{What Remains to be Done?}
% After you have defined and delimited your problem – and presented the relevant results found in the literature within this field, you should sum up which parts of the problem that remain to be solved.

%\section{Objectives}
% The main objectives of this Master’s project are
% 1. This is the first objective
% 2. This is the second objective
% 3. This is the third objective
% 4. More objectives
%All objectives shall be stated such that we, after having read the thesis, can see whether or not you have met the objective. “To become familiar with . . . ” is therefore not a suitable objective.

%\section{Limitations}
% In this section you describe the limitations of your study. These may be related to the study object (physical limitations, operational limitations), to the thoroughness of the analysis, and so on.

%\section{Approach}
% Here you should describe the (scientific) approach that you will use to solve the problem and meet your objectives. You should specify the approach for each objective. If there are any ethical problems related to your approach, these should be highlighted and discussed.

%\section{Structure of the Thesis}
% The ordering of the thesis is as follows: ... Following this, the... are covered. ... is then presented followed by ... Lastly, ... are covered in the Appendix.
% Remark: Notice that chapter and section headings shall be written in lowercase, but that all main words should start with a capital letter.
%%=========================================