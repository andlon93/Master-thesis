\acresetall
%%=========================================
\chapter{Introduction}
% The first chapter of a well-structured thesis is always an introduction, setting the scene with background, problem description, objectives, limitations, and then looking ahead to summarise what is in the rest of the report. This is the part that readers look at first—so make sure it hooks them!
The complexity of human language is one of the biggest distinctions between us and other animals \citep{hauser2002faculty}. Many have tried to explain how human language evolved into what it is today, but no one has managed to do so yet \citep{chomsky1986knowledge, pinker1990natural, muller1861theoretical}. In recent years, computational models have been used to get a greater insight into the field of language evolution. The main contributions have come from the computational models used to test the validity of language evolution theories through computer simulations.

Four computational models which have advanced the field of language evolution have been developed by: \citeauthor{gong2011simulating}, who simulated agents that attempted to reach a consensus by building up a compositional language using spatial naming games; \citeauthor{lipowska2011naming}, who made a computational simulation using naming games which analysed the cultural transmission over several generations; \citeauthor{munroe2002learning}, who made a signalling game where the language was an \acp{ann} and evolved by agents trying to achieve a consensus on the state of the objects in the world; and \citeauthor{lekvam2014co}, who used a naming game and an evolutionary algorithm to simulate the co-evolution of language and social networks.

The computational model designed for this thesis is based on the work by \citeauthor{lekvam2014co}. This thesis will attempt to improve upon his work by adding a new \textit{fitness function}, adding a new method for calculating the weights in the social network, slightly alter how agents decide to conduct a conversation, and introducing a new \textit{genome}. The new genome is based on the work by \citeauthor{schwartz1992unniversals} who formulated a theory which attempts to define personalities through $10$ basic human values. His theory will be presented in Chapter~\ref{ch:background}.

\section{Objectives}
The first objective is to investigate the state-of-the-art within the field of simulating language using computational models. The strengths and weaknesses of the computational models investigated are discussed. The second objective is to choose one of the computational models investigated and extend it in an attempt to learn more about how the evolutionary forces affect language evolution. Lastly, the computational model will continue the trend of simulating the co-evolution of language and social networks.

\begin{comment}
    \section{Objectives}
    The objectives of this thesis are:
    
    \begin{centering}
        \begin{enumerate}
            %\item Investigate the state-of-the-art in computational models that simulate language evolution and discuss their strengths and weaknesses.
            \item Investigate the state-of-the-art within the field of simulating language using computational models. The computational models' strengths and weaknesses are discussed.
            \item The computational model presented in \citet{lekvam2014co} will be extended in an attempt to learn more about how the evolutionary forces affect language evolution.
            \item The computational model will continue the trend of simulating the co-evolution of language and social networks.  
        \end{enumerate}
    \end{centering}
\end{comment}

\section{Contribution}
This study will contribute to the field of computational language evolution models from a computer scientists perspective. Some new ideas will be presented that contribute to enhance our understanding of the effect the evolutionary forces has on language evolution and the co-evolution of language and social networks.

\section{Outline of the Thesis}
In order to understand the field of language evolution and how computational models simulating language evolution are designed, necessary theories are presented in Chapter~\ref{ch:background}. This includes theories on language evolution, graph theory, social networks, \acp{ga}, and a theory concerning how to model personalities. In Chapter~\ref{ch:LitteratureStudy}, four recently developed computational models designed by \citeauthor{gong2011simulating}, \citeauthor{lipowska2011naming}, \citeauthor{lekvam2014co}, and \citeauthor{munroe2002learning}, respectively, will be presented and discussed. The design of the computational model used in this study is presented in Chapter~\ref{ch:Methodology}. Chapter~\ref{ch:Results} presents the results of the seven experiments that were performed using the model presented in the previous chapter. Then, in Chapter~\ref{ch:Discussion} the results are discussed. Finally, in Chapter~\ref{ch:Conclusion} a summary and conclusion of the results and recommendations for future work are presented.      


%\citeauthor{gong2011simulating}, \citeauthor{lipowska2011naming}, \citeauthor{lekvam2014co}, and \citeauthor{munroe2002learning} each recently designed a computational model. Those four models are discussed in Chapter~\ref{ch:Chapter4}.
%\section{Background}
% In this section, you should present the problem that you are going to investigate or analyze; why this problem is of interest; what has, so far, been done to solve the problem, and which parts of the problem that remain.

%\section{Problem Formulation}
% You should define your problem in a clear an unambiguous way and explain why this is a problem, why it is of interest—and to whom. It is also important to delimit the problem area.

%\subsection{Literature Survey}
% You should here present the main books and articles that treat problems that are similar to what you are studying. If you, later in your thesis, describe the “state of the art” – with a detailed literature survey, you may just give a very brief survey here (approx. a quarter of a page). If this is the only literature survey, you need to go into more details. An objective of the literature survey is to show the reader that you are familiar with the main literature within your field of research – so that you do not “reinvent the wheel.”

% References to literature can be given in two different ways: 
% • As an explicit reference: It is shown by Lundteigen and Rausand (2008) and partly also by Rausand and Høyland (2004) that ...
% • As an implicit reference: It is shown (e.g., see Rausand and Høyland, 2004, Chap. 4) that ...

% In the example above, we have used “author-year” references, which is the preferred format.

% Remark: Following agreement with your supervisor, you may also refer by numbers, for example, [1]. 

% When you refer to the scientific literature, you should always write in present tense. Example: Rausand and Høyland (2004) show that ...

%\subsection{What Remains to be Done?}
% After you have defined and delimited your problem – and presented the relevant results found in the literature within this field, you should sum up which parts of the problem that remain to be solved.

%\section{Objectives}
% The main objectives of this Master’s project are
% 1. This is the first objective
% 2. This is the second objective
% 3. This is the third objective
% 4. More objectives
%All objectives shall be stated such that we, after having read the thesis, can see whether or not you have met the objective. “To become familiar with . . . ” is therefore not a suitable objective.

%\section{Limitations}
% In this section you describe the limitations of your study. These may be related to the study object (physical limitations, operational limitations), to the thoroughness of the analysis, and so on.

%\section{Approach}
% Here you should describe the (scientific) approach that you will use to solve the problem and meet your objectives. You should specify the approach for each objective. If there are any ethical problems related to your approach, these should be highlighted and discussed.

%\section{Structure of the Thesis}
% The ordering of the thesis is as follows: ... Following this, the... are covered. ... is then presented followed by ... Lastly, ... are covered in the Appendix.
% Remark: Notice that chapter and section headings shall be written in lowercase, but that all main words should start with a capital letter.
%%=========================================