\acresetall
%%=========================================
\chapter{Results}\label{ch:Results}

This chapter will present the results gathered through 7 experiments using the computational model presented in the previous chapter.

\section{Parameters}
Table \ref{tab:params} displays the parameters, the base values used in the main experiment, and a short explanation of each parameter. In a given experiment, the value of a parameter can be assumed to be the value associated with it in table \ref{tab:params}, unless no other value is specified.    

\begin{table}[htbp]
    \centering
    \caption[Table of the parameters in the model.]{Table of the parameters in the model.}\label{tab:params}
    \begin{tabu} to 1.0\textwidth { X[1,c] X[1,c] X[5,p] }%{| c | c | p{0.8\linewidth} |} 
         \hline
         \rowfont\bfseries
         Parameter & Value & Description \\ 
         \hline
         a & 1000 & Number of agents in the population. \\ 
         g & 100 & Total number of generations the simulation will last for. \\

         d & 1 & Number of dialogues per agent per generation. \\

         k & 0.9 & k\% of the population is randomly chosen into the survival selection pool. \\

         n & 0.5 & The n\% fittest agents in the pool survive to the next generation. \\ 

         m & 0.02 & Probability of mutation. \\

         c & 0.5 & The constant which affects how extrovert agents can act. \\
         $\delta$ & 3 & Constant related to the probability of a child conducting its first dialogues with its parents.\\
         
          $\epsilon$ & 0.2 & Probability in the parent selection pool is $(1-\epsilon)$\\ 

          $\alpha$ & 0.2 & Constant related to the fitness function part which concerns an agent's weights.\\

          $\beta$ & 0.12 & Constant related to the fitness function part which concerns an agent's edges. \\

         $\gamma$ & -0.05 & Constant related to the fitness function part which concerns an agent's age. \\ 
         \hline
    \end{tabu}
\end{table}

\section{Experiments}
This section will present seven experiments. The experiments were conducted in the order in which they are presented. The first experiment represents the main results and the foundation for all experiments performed after it. The next two experiments tested what happens when the population increases or decreases. The fourth experiment was performed with an increased amount of dialogues per generation. The results from the third experiment lead to a hypothesis which is tested in the next experiment through lowering the probability of parent-child dialogues. The hypothesis will be presented in the discussion of the results. Experiment six tests how altering the turnover affects the evolutionary processes in this model. The last experiment adds a probability for agents to invent a new word even though their vocabulary is not empty.

\subsection{Experiment 1 - Main results}
The main experiment was conducted with the exact values from table \ref{tab:params}. See figure~\ref{fig:Fitness} for the figures related to this experiment.

The average fitness of the agents greatly increased during the first $40$ generations, and then the average fitness stabilised at about $0.85$. The average degree of the agents also increased a lot during the first $40$ generations, eventually stabilising at an average of about $10$ edges per agent. 

The easiest method for a newborn agent to acquire connections was through dialogues with its parents, and the probability of that happening was slowly increasing during the entire simulation. The learning rate increased a lot during the first $20$ generations. After the first $20$ generations, the learning rate slowly increased during the remainder of the simulation. The extrovert probability increased during the first three generations, and then decreased to about $5\%$, indicating that agents who focused on gaining strong relationships gained a better fitness than extrovert agents with many weak relationships. 

The graphs describing the percentage of successful dialogues, number of unique highest ranked words in the vocabularies, and average number of words in a vocabulary, all tell the same story. In the beginning there were many words, all agents had large vocabularies, and few dialogues were successful. The number of unique words decreased substantially during the first $10$ generations, then at generation $62$, the population reached consensus. Consensus means that all agents had the same word as their highest weighted word. The graph which display the percentage of successful dialogues quickly reached $70\%$, then percentage gradually rose for the remainder of the simulation, ending at $90\%$. The average vocabulary size increased during the first generations, and then decreased until generation $60$, ending at an average of slightly more than one.

The figures displaying the social network at generation $5$, $20$, $40$ and $100$ only include the agents with a connection, which explains why the number of nodes is not equal in all three figures, and why there is a discrepancy between the social network and the graph displaying the average degree per agent. The farther an agent is away from the centre of the social network, the fewer edges it has. The nodes closest to the centre are the ones with the most edges. At generation $5$, most of the nodes seemed to have two or three connections each. By generation $20$, the number of unique words had gone down and many of the agents had acquired several connections. The network at generation $20$, seemed to be more polarised than it was at generation $5$. Some agents had many connections, while a lot of agents had only one or two connections. At generation $40$, the social network had become even more polarised. The agents in the centre had gotten more connections, while the agents outside of the centre still had one or two edges. At generation 100 most of the population with connections had several connections, as can be seen by the high density of edges at the centre of the figure. There were still a select few agents with one or two edges, but those were by far in a minority.

\begin{figure}[htbp]
    \centering
    \subfigure[The average fitness.]{\includegraphics[width=0.49\linewidth]{fig/Results/Exp1/Fitness1}}
    \hfill
    \subfigure[The average degree.]{\includegraphics[width=0.49\linewidth]{fig/Results/Exp1/Degree1}}
    \par \bigskip
    \subfigure[Number of unique highest ranked words in the population.]{\includegraphics[width=0.49\linewidth]{fig/Results/Exp1/UniqueWords1}}
    \hfill
    \subfigure[Green: Average vocabulary size. Blue: Successful dialogues divided by total number of dialogues.]{\includegraphics[width=0.49\linewidth]{fig/Results/Exp1/Vocabulary1}}
    \par \bigskip
    \subfigure[Green: Average probability of conducting parent-child dialogues. Blue: Average learning rate. Red: Average probability of acting extrovertly.]{\includegraphics[width=0.49\linewidth]{fig/Results/Exp1/Genes1}\label{fig:Genes1}}
    \caption{The graphs from experiment 1}
    \label{fig:Fitness}
\end{figure}
\begin{figure}[htbp]
    \centering
    \ContinuedFloat
    \subfigure[Social network at generation 5.]{\includegraphics[width=0.49\linewidth]{fig/Results/Exp1/_graph5}}
    \hfill
    \subfigure[Social network at generation 20.]{\includegraphics[width=0.49\linewidth]{fig/Results/Exp1/_graph20}}
    \par \bigskip
    \subfigure[Social network at generation 40.]{\includegraphics[width=0.49\linewidth]{fig/Results/Exp1/_graph40}}
    \hfill
    \subfigure[Social network at generation 100.]{\includegraphics[width=0.49\linewidth]{fig/Results/Exp1/_graph100}}
    \caption{Snapshots of the social network of experiment 1 at generation 5, 20, 40, and 100.}
    \label{fig:SN}
\end{figure}


\subsection{Experiment 2 - Small population}
This experiment was conducted with a population of 100. See figure~\ref{fig:Fitness2} for the figures related to this experiment. The variance seems to be higher with fewer agents, causing the average fitness and degree vary more from one generation to another. The average fitness seems to stabilise at about $0.85$ after generation $30$, which was earlier than in the main experiment. The percentage of successful reached $90\%$ after only $15$ generations, and the number of unique highest weighted words reached one after $15$ generations as well. At generation $5$, the social network consisted of a few agents with two or three edges per agent. Up until generation $40$, the social network became more and more dense for each generation that went by. Between generation $40$ and $100$, there seemed to be little change in the structure of the social network.   
\todo{flytte figur genes til appendix}
\begin{figure}[htbp]
    \centering
    \subfigure[The average fitness.]{\includegraphics[width=0.49\linewidth]{fig/Results/Exp2/Fitness1}}
    \hfill
    \subfigure[The average degree.]{\includegraphics[width=0.49\linewidth]{fig/Results/Exp2/Degree1}}
    \par \bigskip
    \subfigure[Number of unique highest ranked words in the population.]{\includegraphics[width=0.49\linewidth]{fig/Results/Exp2/UniqueWords1}}
    \hfill
    \subfigure[Green: Average vocabulary size. Blue: Successful dialogues divided by total number of dialogues.]{\includegraphics[width=0.49\linewidth]{fig/Results/Exp2/Vocabulary1}}
    \par \bigskip
    \subfigure[Green: Average probability of conducting parent-child dialogues. Blue: Average learning rate. Red: Average probability of acting extrovertly.]{\includegraphics[width=0.49\linewidth]{fig/Results/Exp2/Genes1}}
    \caption{The graphs from experiment 2}
    \label{fig:Fitness2}
\end{figure}
\begin{figure}[htbp]
    \centering
    \ContinuedFloat
    \subfigure[The social network at generation 5.]{\includegraphics[width=0.49\linewidth]{fig/Results/Exp2/_graph5}}
    \hfill
    \subfigure[The social network at generation 20.]{\includegraphics[width=0.49\linewidth]{fig/Results/Exp2/_graph20}}
    \subfigure[The social network at generation 40.]{\includegraphics[width=0.49\linewidth]{fig/Results/Exp2/_graph40}}
    \hfill
    \subfigure[The social network at generation 100.]{\includegraphics[width=0.49\linewidth]{fig/Results/Exp2/_graph100}}
    \caption{Snapshots of the social network of experiment 2 at generation 5, 20, 40, and 100.}
    \label{fig:SN2}
\end{figure}

\subsection{Experiment 3 - Large population}
This experiment was performed with a population of $10000$. See figure~\ref{fig:Fitness3} for the figures related to this experiment. The results show that when there were many agents, they required more time to reach consensus. In fact, after 100 generations, the agents had not reached an agreement upon one word. This lead to communication more often being unsuccessful, which again lead to the average fitness and degree being lower than in the results from the main experiment. The difference in the social network is that the agents were divided into one main group and several other groups of two to three agents for the first generations.
\todo{flytte genes til appendix}
\begin{figure}[htbp]
    \centering
    \subfigure[The average fitness.]{\includegraphics[width=0.49\linewidth]{fig/Results/Exp3/Fitness1}}
    \hfill
    \subfigure[The average degree.]{\includegraphics[width=0.49\linewidth]{fig/Results/Exp3/Degree1}}
    \par \bigskip
    \subfigure[Number of unique highest ranked words in the population.]{\includegraphics[width=0.49\linewidth]{fig/Results/Exp3/UniqueWords1}}
    \hfill
    \subfigure[Green: Average vocabulary size. Blue: Successful dialogues divided by total number of dialogues.]{\includegraphics[width=0.49\linewidth]{fig/Results/Exp3/Vocabulary1}}
    \par \bigskip
    \subfigure[Green: Average probability of conducting parent-child dialogues. Blue: Average learning rate. Red: Average probability of acting extrovertly.]{\includegraphics[width=0.49\linewidth]{fig/Results/Exp3/Genes1}}
    \caption{The graphs from experiment 3.}
    \label{fig:Fitness3}
\end{figure}
\begin{figure}[htbp]
    \centering
    \ContinuedFloat
    \subfigure[Social network at generation 5.]{\includegraphics[width=0.49\linewidth]{fig/Results/Exp3/_graph5}}
    \hfill
    \subfigure[Social network at generation 10.]{\includegraphics[width=0.49\linewidth]{fig/Results/Exp3/_graph10}}
    \par \bigskip
    \subfigure[Social network at generation 40.]{\includegraphics[width=0.49\linewidth]{fig/Results/Exp3/_graph40}}
    \hfill
    \subfigure[Social network at generation 100.]{\includegraphics[width=0.49\linewidth]{fig/Results/Exp3/_graph100}}
    \caption{Snapshots of the social network of experiment 3 at generation 5, 10, 40, and 100.}
    \label{fig:SN3}
\end{figure}

\subsection{Experiment 4 - Dialogues}
This experiment was conducted with $d = 5$, meaning that each generation conducted $5$ times as many dialogues as the main experiment did. See figure~\ref{fig:Fitness4} for the figures related to this experiment. The average degree increased when $5$ times as many dialogues were conducted. The increase in dialogues ensures that each agent has more and stronger connections leading to a higher average fitness. the simulation reached consensus at about the same time as the main experiment. The social network evolved similarly as in the main experiment, except that the agents on average had more edges.
\todo{flytte vocabulary, genes, and SN til appendix.}
\begin{figure}[htbp]
    \centering
    \subfigure[The average fitness.]{\includegraphics[width=0.49\linewidth]{fig/Results/Exp4/Fitness1}}
    \hfill
    \subfigure[The average degree.]{\includegraphics[width=0.49\linewidth]{fig/Results/Exp4/Degree1}}
    \par \bigskip
    \subfigure[Number of unique highest ranked words in the population.]{\includegraphics[width=0.49\linewidth]{fig/Results/Exp4/UniqueWords1}}
    \hfill
    \subfigure[Green: Average vocabulary size. Blue: Successful dialogues divided by total number of dialogues.]{\includegraphics[width=0.49\linewidth]{fig/Results/Exp4/Vocabulary1}}
    \par \bigskip
    \subfigure[Green: Average probability of conducting parent-child dialogues. Blue: Average learning rate. Red: Average probability of acting extrovertly.]{\includegraphics[width=0.49\linewidth]{fig/Results/Exp4/genes1}}
    \caption{The graphs from experiment 4.}
    \label{fig:Fitness4}
\end{figure}
\begin{figure}
    \ContinuedFloat
    \subfigure[The social network at generation 5.]{\includegraphics[width=0.49\linewidth]{fig/Results/Exp4/_graph5}}
    \hfill
    \subfigure[The social network at generation 20.]{\includegraphics[width=0.49\linewidth]{fig/Results/Exp4/_graph20}}
    \par \bigskip
    \subfigure[The social network at generation 40.]{\includegraphics[width=0.49\linewidth]{fig/Results/Exp4/_graph40}}
    \hfill
    \subfigure[The social network at generation 100.]{\includegraphics[width=0.49\linewidth]{fig/Results/Exp4/_graph100}}
    \caption{Snapshots of the social network of experiment 4 at generation 5, 20, 40, and 100.}
    \label{fig:SN4}
\end{figure}

\subsection{Experiment 5 - Less Parent-child dialogues}
This experiment was conducted with the constant, $\delta = 1$. $\delta$ is related to the probability of a child having its first dialogues with its parents. See figure~\ref{fig:Fitness5} for the figures related to this experiment. The simulation was ran for $150$ generations, $g = 150$, because it did not reach consensus after $100$ generations, and knowing how this simulation evolved up until consensus might be interesting. The average fitness increased slowly during the entire simulation. The average degree reached four at generation $70$, and during the next $80$ generations the average only increased by $0.5$. The percentage of successful dialogues quickly reached $60\%$, and slowly increased during the entire simulation ending at $93\%$. The average vocabulary size increased during the first $25$ generations, and then decreased and ended at $1.04$. The simulation reached consensus at generation $127$.
\todo{flytte genes og sn til appendix}

\begin{figure}[htbp]
    \centering
    \subfigure[The average fitness.]{\includegraphics[width=0.49\linewidth]{fig/Results/Exp5/Fitness1}}
    \hfill
    \subfigure[The average degree.]{\includegraphics[width=0.49\linewidth]{fig/Results/Exp5/Degree1}}
    \par \bigskip
    \subfigure[Number of unique highest ranked words in the population.]{\includegraphics[width=0.49\linewidth]{fig/Results/Exp5/UniqueWords1}}
    \hfill
    \subfigure[]{\includegraphics[width=0.49\linewidth]{fig/Results/Exp5/Vocabulary1}}
    \caption{The graphs from experiment 5.}
    \label{fig:Fitness5}
\end{figure}
\begin{figure}
    \centering
    \subfigure[Social network at generation 5]{\includegraphics[width=0.49\linewidth]{fig/Results/Exp5/_graph5}\label{exp5SN5}}
    \hfill
    \subfigure[Social network at generation 10]{\includegraphics[width=0.49\linewidth]{fig/Results/Exp5/_graph10}\label{exp5SN10}}
    \par \bigskip
    \subfigure[Social network at generation 20]{\includegraphics[width=0.49\linewidth]{fig/Results/Exp5/_graph20}}
    \hfill
    \subfigure[Social network at generation 150]{\includegraphics[width=0.49\linewidth]{fig/Results/Exp5/_graph150}}
    \caption{Snapshots of the social network of experiment 5 at generation 5, 10, 20, and 150.}
\end{figure}
\subsection{Experiment 6 - Turnover}
The first experiment was performed with $k = 0.95$ and $n = 0.7$. See figure~\ref{fig:Fitness6} for the figures related to this experiment. This simulation was run for $150$ generations because the population were far away from reaching consensus after $100$ generations. The figures that display the genes and the evolution of the social network have not been included because they were similar to the ones from the main experiment. Both the average fitness and degree stabilised after about $75$ generations. The average degree stabilised at $11$ and the average fitness at $0.88$. The number of unique words decreased a lot during the first $20$ generations. The simulation did not reach consensus. It reached two unique words at generation $125$, and ended with two unique words after all $150$ generations.   

The second experiment related to turnover was conducted with $k = 0.95$ and $n = 0.85$. See figure~\ref{fig:6.1} for the figure related to this experiment. The only figure included was the one related to the number of unique words because it is the only one who will be discussed later on. \todo{The other figures can be found in appendix xx}. This simulation was ran for $200$ generations in order for it to reach consensus. Consensus was reached at generation $178$. At generation $125$, when the previous experiment had reached two words, this simulation had six unique words. 

\todo{legge inn genes og sn for 6.0 og alt bortsett fra unique words for 6.1 i appendix}

\begin{figure}[htbp]
    \centering
    \subfigure[The average fitness.]{\includegraphics[width=0.49\linewidth]{fig/Results/Exp6/Fitness1}}
    \hfill
    \subfigure[The average degree.]{\includegraphics[width=0.49\linewidth]{fig/Results/Exp6/Degree1}}
    \par \bigskip
    \subfigure[Number of unique highest ranked words in the population of the first experiment.]{\includegraphics[width=0.49\linewidth]{fig/Results/Exp6/UniqueWords1}}
    \hfill
    \subfigure[Green: Average vocabulary size. Blue: Successful dialogues divided by total number of dialogues.]{\includegraphics[width=0.49\linewidth]{fig/Results/Exp6/Vocabulary1}}
    \par \bigskip
    \subfigure[Number of unique highest ranked words in the population of the second experiment.]{\includegraphics[width=0.49\linewidth]{fig/Results/Exp6.1/UniqueWords1}\label{fig:6.1}}
    \caption{The graphs from experiment 6.}
    \label{fig:Fitness6}
\end{figure}


\subsection{Experiment 7 - making up new words}
This experiment was conducted with an added probability for agents to invent a new word even though they had at least one word in their vocabulary. The probability was set to $0.25$ in this experiment. See figure~\ref{fig:Fitness7} for the figures related to this experiment. The average fitness stabilised at $0.37$, and the average degree stabilised at about $6.5$. Both of these values were considerably lower than in the main experiment. The average vocabulary size was $6$ and the percentage of successful dialogues quickly reached $0.5$ but it did not slowly increase like in the previous experiments. The number of unique highest ranked words evolved similarly to the main experiment. The population reached consensus at generation $68$, only six generations after the main experiment. 

One other experiment was performed with the probability set to $0.45$. See figure~\ref{fig:7.1} for the figure related to this experiment. The other graphs have not been included because they will not be relevant to the discussion\todo{legg ved de andre grafene i appendix}. The number of unique words quickly dropped to about five, but from five it struggled to actually reach consensus. From generation $61$ to the end of the simulation, the number of unique highest ranked words varied from one to three.  
\todo{flytte genes til appendix}
\begin{figure}[htbp]
    \centering
    %Figurtekst, tror jeg. Men i følge Randi skal sånn stå i hovedfigurteksten: Figurer fra resultater fra eksp. 7: a) Avg. fitness; b) avg. degree; and c) ....
    \subfigure[The average fitness.]{\includegraphics[width=0.49\linewidth]{fig/Results/Exp7/Fitness1}}
    \hfill
    \subfigure[The average degree.]{\includegraphics[width=0.49\linewidth]{fig/Results/Exp7/Degree1}}
    \par\bigskip
    \subfigure[Number of unique highest ranked words in the population of the first experiment.]{\includegraphics[width=0.49\linewidth]{fig/Results/Exp7/UniqueWords1}}
    \hfill
    \subfigure[Green: Average vocabulary size. Blue: Successful dialogues divided by total number of dialogues.]{\includegraphics[width=0.49\linewidth]{fig/Results/Exp7/Vocabulary1}}
    \par\bigskip
    \subfigure[Number of unique highest ranked words in the population of the second experiment.]{\includegraphics[width=0.49\linewidth]{fig/Results/Exp7.1/UniqueWords1}\label{fig:7.1}}
    \caption{The graphs from experiment 7.}
    \label{fig:Fitness7}
\end{figure}