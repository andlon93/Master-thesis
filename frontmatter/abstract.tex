%%=========================================
\cleardoublepage
\selectlanguage{english}
\begin{abstract}
Scientists have been researching how human language evolved into the complex language it is today for a long time. Understanding how human language has evolved is a difficult task because there has not been discovered any historical data concerning the earliest forms of human communication. In recent years, computer scientists have attempted to simulate language evolution using computational models. 

This thesis will continue the work of using computer simulations to investigate the effects cultural, biological, and social evolution has on language evolution. Four state-of-the-art computational models are discussed in this thesis \citep{lipowska2011naming, lekvam2014co, gong2011simulating, munroe2002learning}. Theories that are needed to understand the models are also presented. 

The computational model designed for this thesis is an extension of \citeauthor{lekvam2014co}'s work. In order to further investigate the effects cultural and biological evolution the model used in thesis had a redesigned fitness function, method of calculating the weight of a word, and genome. The genome was designed based on a theory, which was formulated by \citeauthor{schwartz1992unniversals}, called the theory of $10$ basic values \citep{schwartz1992unniversals}. This theory claims that a person's personality is defined by the values that person has.  

The experiments show that parts of the computational model worked as intended. The results indicate that mostly acting introvertly is beneficial compared to an extrovert strategy, that the first years are crucial when it comes to an agent's language, and that language evolution to a certain extent is affected by the Baldwin effect. However, there are some simplifications that simply can not be ignored. The naming game is an overly simplification of human language and that might lead the simulations to reach one common language too quickly. The social network does not properly depict the evolution of a social structure because there are no geographical distances between the individuals. This leads to the social network becoming one large network too quickly. One large social network aided the simulation in reaching one common language quickly.

This research has contributed with some new ideas that seem to work as intended, and it has brought new issues into the light. With future work being put into researching language games and how social networks affect language evolution, it could bring us one step closer to understanding how human language evolved.
\end{abstract}
%%=========================================
%%=========================================
\cleardoublepage
\selectlanguage{norsk}
\begin{abstract}
Det har lenge blitt forsket på hvordan det menneskelige språk utviklet seg til det komplekse språket det er i dag. Det finnes ikke historisk data fra de tidligste fasene av det menneskelige språk, som gjør det veldig veldig vanskelig å forstå hvordan det menneskelige språk oppstod og utviklet seg i begynnelsen. Nylig har det blitt forøkt å simulere språkutvikling ved bruk av datamodeller.

Denne oppgaven fortsetter arbeidet med å bruke datasimuleringer til å undersøke effekten kulturell, biologisk og sosial evolusjon har på språkutvikling. Fire nylig designede datamodeller vil diskuteres, og teorien som kreves for å forstå modellene vil presenteres i denne oppgaven \citep{lipowska2011naming, lekvam2014co, gong2011simulating, munroe2002learning}. 

Datamodellen som ble designet til denne oppgaven er inspirert av arbeidet til \citeauthor{lekvam2014co}. Får å undersøke effekten av kulturell og biologisk evolusjon videre ble modellen redesignet med en ny \textit{fitness function}, en ny metode for å kalkulere vekten av et ord og et nytt genom. Genomet ble designet basert på arebidet til \citeauthor{schwartz1992unniversals} som definerte en teori kalt \textit{the theory of $10$ basic values}. Teorien påstår at personligheter kan defineres utifra hvilke verdier en person har.

Eksperimentene viser at deler av modellen fungerte som planlagt. Resultatene indikerer at en introvert strategi er ner lønnsom enn en ekstrovert strategi, at de første årene er avgjørende for en agents' språk og at Baldwin effekten påvirker utviklingen av språk. Resultatene viste også at deler av modellen var for simplifiserte. \textit{Naming gamet} er en kraftig forenkling av det menneskelige språk og det kan ha ført til at simuleringer nådde et felles språk for raskt. Det sosiale nettverket tok ikke hensyn til geografiske avstander, som kan ha ført til at det sosiale nettverket ble til et stort nettverk for fort. At alle agentene var en del av samme nettverk har også bidratt til at det gikk raskere å nå et felles språk.

Forskningen har bidratt med flere gode ideer som virker å fungere. Den har brakt frem flere nye problemstillinger som må studeres videre. Videre arbeid innen språkmodeller og hvordan sosiale nettverk påvirker utviklingen av språk kan bringe oss et steg nærmere å forstå hvordan det menneskelige språk har utviklet seg til det det er i dag. 
\end{abstract}
\selectlanguage{english}
%%=========================================
% Here you give a summary of your work and your resuls. This is like a management summary and should be written in a clear and easy language, without many difficult terms and without abbreviations. Everything you present here must be treated in more detail in the main report. You should not give any references to the report in the summary - just explain what you have done and what you have found out. Should NOT be more than two pages!
\begin{comment}
The results from this model indicate that it partly simulates language evolution successfully. The model indicates that the first years are of great importance for an individual's language, that mostly acting introvertly is beneficial, and that the Baldwin effect has an effect on language evolution. However, some parts of the model were made too simplistic to simulate language evolution in a sufficiently realistic manner. These weaknesses will have to be improved upon in order to learn more about how human language has evolved into what it is today.


 
The fitness function implemented seemed to fulfil its purpose, both extrovert and introvert strategies had advantages in this fitness function. The simulations preferred an extrovert strategy during the initial generations, but in the end, an introvert strategy was favoured. In real life, the vast majority of conversations a person conducts are with others that person already knows, meaning that a introvert strategy is favoured.
 
Parent-child dialogues were the most important factor in order for the simulations to reach a consensus. This indicates that the initial dialogues a newborn conducts were of crucial importance for its language, and it was virtually impossible to learn a new language properly after having lived for two generations. This result indicates that being taught the same languages during the early years is an important factor on the \textit{ontogenetic} time scale. Both in this model and in reality, language is easier to learn while one is young and the older one becomes the harder it is to learn new languages.
 
The more the selection pressure was reduced, the more the simulations struggled to reach a consensus. When the selection pressure was reduced, the effects of the \textit{phylogenetic} (Biological evolution) and \textit{glossogenetic} (Cultural evolution) time scales were reduced. This indicates that these two evolutionary forces have an impact on language evolution. It also indicates that the Baldwin effect \citep{baldwin1896new} affects the cultural and biological forces of language evolution. Other simulations have also concluded that the Baldwin effect has an effect on language evolution \citep{lipowska2011naming, zollman2010plasticity, chater2010language}.

The results from this model indicate that it partly simulates language evolution successfully. The model indicates that the first years are of great importance for an individual's language, that mostly acting introvertly is beneficial, and that the Baldwin effect has an effect on language evolution. However, some parts of the model were made too simplistic to simulate language evolution in a sufficiently realistic manner. These weaknesses will have to be improved upon in order to learn more about how human language has evolved into what it is today.
\end{comment}