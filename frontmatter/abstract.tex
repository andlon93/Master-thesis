%%=========================================
\cleardoublepage
\selectlanguage{english}
\begin{abstract}
Scientists have been researching how human language evolved into the complex language it is today for a long time. Understanding how human language has evolved is a difficult task because there has not been discovered any historical data concerning the earliest forms of human communication. In recent years, computer scientists have attempted to simulate language evolution using computational models. 

This thesis will continue the work of using computer simulations to investigate the effects cultural, biological, and social evolution has on language evolution. Four state-of-the-art computational models are discussed in this thesis \citep{lipowska2011naming, lekvam2014co, gong2011simulating, munroe2002learning}. Theories that are needed to understand the models are also presented. 

The computational model presented in this thesis is an extension of \citeauthor{lekvam2014co}'s work. In order to further investigate the effects cultural and biological evolution the model used in thesis had a redesigned fitness function, method of calculating the weight of a word, and genome. The genome was designed based on a theory, which was formulated by \citeauthor{schwartz1992unniversals}, called the theory of $10$ basic values \citep{schwartz1992unniversals}. This theory claims that a person's personality is defined by the values that person has.  

The experiments show that parts of the computational model worked as intended. The results indicate that mostly acting introvertly is beneficial compared to an extrovert strategy, that the first years are crucial when it comes to an agent's language, and that language evolution to a certain extent is affected by the Baldwin effect. However, there are some simplifications that simply can not be ignored. The naming game is an overly simplification of human language and that might lead the simulations to reach one common language too quickly. The social network does not properly depict the evolution of a social structure because there are no geographical distances between the individuals. This leads to the social network becoming one large network too quickly.

This research has contributed with some new well-functioning ideas, and brought new issues out into the light. With future work being put into researching language games and how social networks affect language evolution could bring us one step closer to understanding how human language evolved.
\end{abstract}
%%=========================================
%%=========================================
\cleardoublepage
\selectlanguage{norsk}
\begin{abstract}



\end{abstract}
\selectlanguage{english}
%%=========================================
% Here you give a summary of your work and your resuls. This is like a management summary and should be written in a clear and easy language, without many difficult terms and without abbreviations. Everything you present here must be treated in more detail in the main report. You should not give any references to the report in the summary - just explain what you have done and what you have found out. Should NOT be more than two pages!
\begin{comment}
The results from this model indicate that it partly simulates language evolution successfully. The model indicates that the first years are of great importance for an individual's language, that mostly acting introvertly is beneficial, and that the Baldwin effect has an effect on language evolution. However, some parts of the model were made too simplistic to simulate language evolution in a sufficiently realistic manner. These weaknesses will have to be improved upon in order to learn more about how human language has evolved into what it is today.


 
The fitness function implemented seemed to fulfil its purpose, both extrovert and introvert strategies had advantages in this fitness function. The simulations preferred an extrovert strategy during the initial generations, but in the end, an introvert strategy was favoured. In real life, the vast majority of conversations a person conducts are with others that person already knows, meaning that a introvert strategy is favoured.
 
Parent-child dialogues were the most important factor in order for the simulations to reach a consensus. This indicates that the initial dialogues a newborn conducts were of crucial importance for its language, and it was virtually impossible to learn a new language properly after having lived for two generations. This result indicates that being taught the same languages during the early years is an important factor on the \textit{ontogenetic} time scale. Both in this model and in reality, language is easier to learn while one is young and the older one becomes the harder it is to learn new languages.
 
The more the selection pressure was reduced, the more the simulations struggled to reach a consensus. When the selection pressure was reduced, the effects of the \textit{phylogenetic} (Biological evolution) and \textit{glossogenetic} (Cultural evolution) time scales were reduced. This indicates that these two evolutionary forces have an impact on language evolution. It also indicates that the Baldwin effect \citep{baldwin1896new} affects the cultural and biological forces of language evolution. Other simulations have also concluded that the Baldwin effect has an effect on language evolution \citep{lipowska2011naming, zollman2010plasticity, chater2010language}.

The results from this model indicate that it partly simulates language evolution successfully. The model indicates that the first years are of great importance for an individual's language, that mostly acting introvertly is beneficial, and that the Baldwin effect has an effect on language evolution. However, some parts of the model were made too simplistic to simulate language evolution in a sufficiently realistic manner. These weaknesses will have to be improved upon in order to learn more about how human language has evolved into what it is today.
\end{comment}