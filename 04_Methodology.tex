\acresetall
%%=========================================
\chapter{Methodology}\label{ch:Methodology}

\section{Discussion}
The social networks used in the models presented in the previous chapter were relatively simple. For example, the fitness functions in the previous models highly reward an extrovert strategy, i.e. attempting to meet new people. But in reality, the majority of conversations in most peoples' lives are conducted with people they already know. A fitness function should ideally allow both strategies to be viable and allow the evolutionary process to display which traits are favourable.

It has also been assumed that if two agents with the same language conduct a conversation, the conversation was always viewed as a success, and the agents would develop a strong connection over time. However, many people with the same language can not connect because of personal differences and not because they are unable to communicate. If a person does not fit into a group, that person might look for a group that fits the person better, or slowly alter his/hers personality in order to fit in. Cultural evolution is about how language was transferred between individuals, and between generations. 

How language is transferred between individuals is affected by whom one chooses to converse with. How someone decides to interact with others might be affected by that someone’s personality. A social network can be used to visualise how and with whom someone chooses to interact. Personality might have an effect on a person’s social network, and incorporating individual personality into a language game based model and visualising it through a social network might enhance the understanding of the effects cultural learning has on language evolution. 

Among the models studied, Lekvam’s model was best suited for these extensions because it already had a social network implemented. The model also use a naming game, which is a language game that is often used when the effects of cultural learning are studied. Lipowska’s model used a lattice, which does not suit the proposed extension because it does not allow the number of connections an agent has to change. Gong’s and Munroe \& Cangelosi’s models primarily studied compositional language and not the effects of social networks. Also, Lekvam argued that his model was explicitly made in such a manner that it was suited for extensions.

\section{overskrift?}

This computational model is based on Lekvam’s model, adding personalities based on Schwartz' 10 values model \citep{lekvam2014co, schwartz2012overview}. This chapter will also present a new method for choosing a listener, calculating the weights in the social network, and a new fitness function.

The model uses a genetic algorithm which simulates the evolution of language and social networks over time. Tournament selection is used as parent selection. In tournament selection, a pool of random agents is chosen. Then the agents are sorted based on their fitness. The probability of an agent being chosen is:
\begin{equation}
    \label{eq:TSelect}
    P_{parent} = p(p-1)^{i}
\end{equation}
where $i$ is a counter starting at zero, and $p$ equals the probability of the fittest agent in the pool being chosen as the parent. Turnover selection is performed by randomly choosing $k\%$ of the population into a pool. Then, the $n\%$ fittest agents of that pool survive to the next generation. 

Personalities are added into the model in the form of an agent’s genome. The genotype of an agent consists of one number $g$, $0 \leq g \geq 100$, for each of the ten values from Schwartz’s model. The values can not be randomly initiated because some of the values describe relatively equal goals, while some other values are contradictory. If two contradicting values both have a high value, the goals of that person would contradict each other. So, in order to ensure the validity of the relations between the ten values, not all values can be created randomly. In figure \ref{fig:TenValuesSchwartz}, the values are divided into four groups. Values within a group are related and should have numerical values close to each other, while values on the opposite side of the circle should have opposite numerical values. In this model, the validity of the relations between the values is kept by randomly setting the self-direction and achievement value. Values in the same quadrant are set by using this formula:

\begin{equation}\label{eq:valuesInSameQuadrant}
V_{i} = \frac{\mathrm{random}(0, 20)}{100} \ast V_{i-1},
\end{equation}
where $V_{i}$ is the number associated with value $i$, $V_{i-1}$ is the number associated with the value counter clockwise of $V_i$ in the circle, and $\mathrm{random}(0, 20)$ is a random integer $i$, $0 \leq i \geq 20$. Security and benevolence is set based on the opposite value using the formula:
\begin{equation}\label{eq:valuesInOppositeQuadrant}
V_{i} = \frac{\mathrm{random}(1, 20)}{100} \ast (100 - V_{\mathrm{opposite}}).
\end{equation}

The other values in the quadrants with security or benevolence are set based on the formula \ref{eq:valuesInSameQuadrant}. Hedonism is split between two quadrants and it is therefore calculated based on the two values next to it in figure \ref{fig:TenValuesSchwartz} through this formula:
\begin{equation}\label{eq:hedonism}
V_{i} = \frac{\mathrm{random}(1, 20)}{100} \ast \frac{V_{i-1} + V_{i+1}}{2}.
\end{equation}

The values of the genome is then normalised, representing the individual’s personality and phenotype. A newborn’s genotype is made using crossover from two parents' genotypes, and the new genotype has a small probability of mutation. Mutation is conducted by randomly altering one value in the genotype. Each agent has a vocabulary similar to the vocabulary used in Lekvam’s model. The vocabulary consists of the utterances an agent has heard or created. Each utterance has a corresponding weight describing how successful the utterance has been in past conversations.

The social network in the model is represented by a weighted discrete graph, where the vertices represent the agents, and the edges represent the connections between the agents. If a dialogue is successful, the weight of the edge between the two agents increases and if it was unsuccessful the weight decreases. An edge gets added a weight of $1$ if the dialogue is successful and $0.5$ is subtracted if the dialogue was unsuccessful. This method of calculating the weight allows the language of the agent to affect the weight of the edge.
 
The fitness function used is made to represent how well an agent communicates with the members of the community, as well as how the agent’s personality fits in with the group. An agent's ability to communicate was projected into the social network because the weights were adjusted based on an agent’s ability to be understood. The personality of the agents determines how good it is at learning a language, how extrovert it is, and whether it cares about its children. The fitness function also takes into account the age of the agent. The fitness function is.

\begin{equation}\label{eq:Fitness}
\text{Fitness(j)} = \Big( e^{\alpha \sum_{i=1}^{N}{W_{ji}}}-1 \Big) \ast \Big( e^{\beta N_{W_{ji \geq \Theta}}}-1 \Big) \ast \gamma e^{-0.05 \text{age}}
\end{equation}

where $W_{ji}$ is the weight of the connection between agent j and agent i, $N$ is the total number of connections of agent j, $ N_{W_{ji \geq \Theta}}$ is the total number connections agent j that are stronger than the threshold $\Theta$, and $\alpha$, $\beta$, and $\gamma$ are constants adjusting the three parts of the fitness function. The first part calculates the sum of all weights of an agent to display how strongly it connects to others. The second part calculates how many agents with a connection strength stronger than the threshold, $\Theta = 0.5$, agent j has. The third part provides an agent with less fitness the older it becomes. These three functions combined display the agent’s fitness in the society. They take into account the number of connections it has, the strength of the connections, and its age.

When an agent chooses a listener, the agent's social network and personality are used. An agent can either be extrovert or introvert, based on its values. The positive extrovert value is achievement; people often seek recognition for their achievements. Many people enjoy sharing achievements with others. For some, the achievement has greater value if many knows about it. The negative value is conformity; restraint from doing certain actions because one is scared of the consequences. People who value this high think a lot about consequences and might be careful in social settings because of it. Each of these values are taken from an agent’s genome and is therefore in the interval $[0, 100]$. Based on these values, the following formula describes the probability of an agent reaching out to agents that it does not already have a connection with

\begin{equation}\label{eq:chooseListener}
P = \parentheses{\frac{\sum \text{extrovert values} + (100 - \sum \text{introvert values})}{200}C}/100
\end{equation}

where $C$ is a constant, between $0 < C \leq 1$, which forces all individuals to have a tendency towards introvertedness. People tend to have more conversations with people they already know, so skewing the probability towards introvertedness seems reasonable. The first conversations a newborn has might be with its parents. The probability of having dialogues with parents is equal to the normalised value of the security value in the genome. Each parent will conduct three dialogues each and the newborn will function as the listener in all these conversations. If the speaking agent $i$ decides to act introvertly, the probability of contacting one specific agent $j$ is related to the weight between the agents:
\begin{equation}\label{eq:intrvertListener}
P_{i,j} = \frac{W_{ij}}{\sum_{k=1}^{N}{W_{ik}}},
\end{equation}
Where $N$ is the number of edges the agent has, and $W_{ij}$ is the weight of an edge. If the agent decides to be extrovert, it randomly chooses an agent from the population to contact.

Agents also alter their personality based on who they communicate with. After a conversation, both agents alter their values using this formula:

\begin{equation}\label{eq:PCR}
\text{Personality change rate} = \parentheses{\frac{\Delta \textit{Value} \ast \textit{isSuccess}}{\text{age}}}\ast C,
\end{equation}

where $\Delta \textit{Value}$ is the difference of a specific value in the two agents' genomes, $\textit{isSuccess}$ is a boolean which ensures that the personality is not changed if a conversation is unsuccessful, $\text{age}$ is the number of generations the agent has lived, and $C$ is a constant, $0 < C \geq 1$. $C$ makes the personality change rate have less of an impact the the smaller it is. After this is calculated for all values in the genome, it is normalised again.

One conversation is conducted by the speaker uttering a word $i$ from its vocabulary. The probability of a specific word from the vocabulary being uttered is:

\begin{equation}\label{eq:chooseWordprob}
P_{i} = \frac{W_{i}}{\sum_{j=1}^{N}W_{j}}
\end{equation}

where $W_{i}$ is the weight of word $i$ in the speaking agent's vocabulary, and $\sum_{j=1}^{N}W_{j}$ is the sum of the weights of all words in its vocabulary. If the speaking agent has an empty vocabulary, it invents a new word. 

If the listener has the utterance in its vocabulary, the conversation is deemed as a success, and the weight of the edge between the agents in the social network is increased, the personalities of both the listener and speaker are altered, and the weight of the word in the vocabularies of both agents are increased. If the listener did not have the word in its vocabulary, the conversation is deemed a failure, and the weight of the edge between the agents in the social network decreases. The speaking agent decreases the weight of the word in its vocabulary, while the listener adds the word into its vocabulary. None of the agents' personalities are altered when the conversation is unsuccessful.

%%=========================================