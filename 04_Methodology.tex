\acresetall
%%=========================================
\chapter{Methodology}\label{ch:Methodology}

The social networks used in the models presented in the previous chapter were relatively simple. For example, the fitness functions in the previous models highly reward an extrovert strategy, i.e. attempting to meet new people. However, in reality, the majority of conversations in most peoples' lives are conducted with people they already know. A fitness function should ideally allow both strategies to be viable and allow the evolutionary process to display which traits are favourable.

The previously discussed models also assumed that if two agents with the same language conduct a conversation, the conversation was always viewed as a success, and the agents would develop a strong connection over time. However, many people with the same language can not connect with each other due to personal differences, even though they are able to communicate. If an agent does not fit into a group, the agent has two main strategies to improve its situation. Either, the agent can look for another group that is more compatible with the agent's personality, or the agent can slowly alter its personality to fit its original group. 

Cultural evolution is about the transferring of language between individuals between individuals and between generations. In what way language is transferred between individuals is affected by whom one chooses to converse with. How someone decides to interact with others might be affected by that someone’s personality. A social network can be used to visualise how and with whom someone chooses to interact. Personality might have an effect on a person’s social network, and incorporating individual personality into a language game based model and visualising it through a social network might enhance the understanding of the effects cultural learning has on language evolution. 

Among the models studied, Lekvam’s model was best suited for these extensions because it already had a social network implemented. The model also use a naming game, which is a language game that is often used when the effects of cultural learning are studied \citep{lekvam2014co, lipowska2011naming, steels2011modeling}. Lipowska’s model used a lattice, which does not suit the proposed extension because it does not allow the number of connections an agent has to change. The models made by \citeauthor{gong2011simulating} and \citeauthor{munroe2002learning} primarily studied compositional language and not the effects of social networks. Also, Lekvam argued that his model was explicitly made in such a manner that it was suited for extensions.

\section{Genetic Algorithm}

The computational model developed during this study is based on Lekvam’s model. As opposed to his model, this model add personalities based on Schwartz' 10 values model \citep{lekvam2014co, schwartz2012overview} in order to further investigate the understanding of the effect the evolutionary forces have on language evolution. The model will also present a new method for choosing a listener, calculating the weights in the social network, and evaluating the fitness of the agents.

The model uses a genetic algorithm which simulates the evolution of language and social networks over time. Parent selection is performed using tournament selection. First, a pool of agents is randomly chosen. Then, these agents are sorted from highest to lowest fitness and given a value corresponding to their ranking, i.e. the agent with the highest fitness has rank 0. Finally, the probability of an agent being chosen to become a parent is
\begin{equation}
    \label{eq:TSelect}
    P_{\mathrm{parent}} = p(p-1)^{i}
\end{equation}
where $i$ is the agent's rank in the pool and $p$ equals the probability of the fittest agent in the pool being chosen as the parent. A newborn agent's genotype is made from the genotypes of its two parents using crossover. The newborn agent's genotype has a small probability of mutation, which is conducted by randomly altering one value in the genotype.

At the end of each generation, turnover selection is performed by randomly choosing $k\%$ of the population into a pool. Then, the $n\%$ fittest agents of that pool survive to the next generation.

\section{Personalities}

Personalities are added into the model in the form of an agent’s genome. The genotype of an agent consists of ten numbers, in the interval between 0 and 100, each representing one of the ten values from Schwartz' model of basic human values. The values cannot be randomly initiated because some of the values depend on each other as some of the values describe relatively equal goals. If two contradicting values both have a high value, the goals of that person would contradict each other. So, in order to ensure the validity of the relations between the ten values, not all values can be created randomly. Schwartz categorises his ten values into four groups, as seen in Figure~\ref{fig:TenValuesSchwartz}. Values within a group are related and should have numerical values close to each other, while values on the opposite side of the circle should have opposite numerical values. In this model, the validity of the relations between the values is kept by randomly setting the self-direction and achievement value. Values in the same quadrant are set by using the following formula

\begin{equation}\label{eq:valuesInSameQuadrant}
V_{i} = \frac{\mathrm{random}(0, 20)}{100} \cdot V_{i-1},
\end{equation}
where $V_{i}$ is the number associated with value $i$, $V_{i-1}$ is the number associated with the value counter clockwise of $V_i$ in the circle, and $\mathrm{random}(0, 20)$ is a random integer $i$, $0 \leq r \leq 20$. Security and benevolence is set based on the opposite value using the formula:
\begin{equation}\label{eq:valuesInOppositeQuadrant}
V_{i} = \frac{\mathrm{random}(1, 20)}{100} \cdot (100 - V_{\mathrm{opposite}}).
\end{equation}

The other values in the quadrants with security or benevolence are set based on Equation~\eqref{eq:valuesInSameQuadrant}. Hedonism is split between two quadrants and it is therefore calculated based on the two values next to it in Figure~\ref{fig:TenValuesSchwartz} through this formula:
\begin{equation}\label{eq:hedonism}
V_{i} = \frac{\mathrm{random}(1, 20)}{100} \cdot \frac{V_{i-1} + V_{i+1}}{2}.
\end{equation}

The values of the genome is then normalised. The resulting genome is representing the individual’s personality and phenotype.  Each agent has a vocabulary similar to the vocabulary used in Lekvam’s model. The vocabulary consists of the utterances an agent has heard or created. Each utterance has a corresponding weight describing how successful the utterance has performed in past conversations.

\section{Fitness Function}

The fitness function is designed to both represent how well an agent communicates with the members of the community, and how the agent’s personality fits in with the group. An agent's ability to communicate was projected into the social network because the weights were adjusted based on an agent’s ability to be understood. The personality of the agents determines how good it is at learning a language, how extrovert it is, and whether it cares about its children. The fitness function also takes into account the age, $t$, of the agent. The fitness function is

\begin{equation}\label{eq:Fitness}
\text{Fitness}(j) = \left[ \exp\left(\alpha \sum_{i=1}^{N}{W_{ij}}\right)-1 \right] \left[ \exp\left(\beta N_{W_{ij} \geq \Theta}\right) -1 \right] \gamma \exp\left(-0.05 t\right)
\end{equation}

where $W_{ij}$ is the weight of the connection between agent $i$ and agent $j$, $t$ is the number of generations it has lived, $N$ is the total number of connections of agent $j$, $ N_{W_{ij} \geq \Theta}$ is the total number of connections that agent $j$ has that are stronger than the threshold $\Theta$, and $\alpha$, $\beta$, and $\gamma$ are constants adjusting the three factors of the fitness function. The first factor takes into account how strongly an agent connects to others by calculating the sum of all its weights. The second factor calculates how many agents with a connection strength stronger than the threshold $\Theta$ agent $j$ has. The third factor provides an agent with less fitness the older it becomes. These three factors combined display the agent’s fitness in the society. They take into account the number of connections the agent has, the strength of the connections, and the age of the agent.

\section{Social Network}

The social network in the model is represented by a weighted discrete graph, where the vertices represent the agents, and the edges represent the connections between the agents. If a dialogue is successful, i.e. the listeners has the uttered word in its vocabulary, the weight of the edge between the two agents is increased. If the dialogue was unsuccessful, i.e. the listener did not have the uttered word in its vocabulary, the weight is decreased. An edge gets added a weight of $1$ if the dialogue is successful and $0.5$ is subtracted if the dialogue was unsuccessful. This method of calculating the weight allows the language of the agent to affect the weight of the edge.

When an agent chooses a listener, the agent takes into account both its social network and its personality. An agent can either be extrovert or introvert, which is reflected in its genome. The extrovert trait is connected to a high achievement value, which is found in people who seek recognition for their achievements. Many people enjoy sharing achievements with others. For some, the achievement has greater value if many knows about it. The introvert trait is connected to a high conformity value, which is found in people who are very aware of the consequences of their actions and might be careful in social settings because of it. Each of these values are taken from an agent’s genome and is therefore between 0 and 100. Based on these values, the following formula describes the probability of an agent reaching out to agents that it does not already have a connection with

\begin{equation}\label{eq:chooseListener}
P = \parentheses{\frac{\sum \text{extrovert values} + (100 - \sum \text{introvert values})}{200}C}/100
\end{equation}
where $C$ is a constant in the range $0 < C \leq 1$, which forces all individuals to have a tendency towards introvertedness. People tend to have more conversations with people they already know, and hence skewing the probability towards introvertedness is reasonable. The first conversations a newborn agent conducts, might be with its parents. The probability of having dialogues with parents is equal to the normalised value of the security value in the genome. Each parent will conduct three dialogues each and the newborn will function as the listener in all these conversations. If the speaking agent $i$ decides to act introvertly, the probability of contacting one specific agent $j$ is related to the weight between the two agents by the following relation
\begin{equation}\label{eq:intrvertListener}
P_{i,j} = \frac{W_{ij}}{\sum_{k=\{\text{connected to } i\}} {W_{ik}}},
\end{equation}
where $W_{ij}$ is the weight of an edge and the summation is performed on all agents connected to agent $i$. If the agent decides to be extrovert, it randomly chooses an agent to contact from the population it is not already connected to. If the parents do not conduct conduct conversations with their newborn agent, the agent will have its for first dialogue with whomever that contacts it first from the population.

\clearpage
Agents also alter their personality based on who they communicate with. An agent's personality is updated by calculating the \ac{pcr} for all the values in the genome. Then each value is updated through the formula $\mathrm{new_{personality}} = \mathrm{old_{personality}} + \mathrm{old_{personality}} \cdot \mathrm{pcr}$. After a conversation, both agents alter their values using the following formula
\begin{equation}\label{eq:PCR}
\text{Personality change rate} = C\frac{\Delta V_i S}{t},
\end{equation}
where $\Delta V_i$ is the difference of a specific value $i$ in the two agents' genomes, $S$ equals one if the conversation was successful and zero if it was unsuccessful, which ensures that the personality is not changed if a conversation is unsuccessful, $t$ is the number of generations the agent has lived, and $C$ is a constant, $0 < C \leq 1$. $C$ makes the \ac{pcr} have less of an impact the smaller $C$ is. After the \ac{pcr} is calculated for all values in the genome, the genome is normalised again.

A conversation is conducted by the speaker uttering a word $i$ from its vocabulary. The probability of a specific word from the vocabulary being uttered is given by
\begin{equation}\label{eq:chooseWordprob}
P_{i} = \frac{W_{i}}{\sum_{j=1}^{N}W_{j}}
\end{equation}
where $W_{i}$ is the weight of word $i$ in the speaking agent's vocabulary, and $\sum_{j=1}^{N}W_{j}$ is the sum of the weights of all words in the agent's vocabulary. If the speaking agent has an empty vocabulary, it invents a new word by randomly deciding the a length of the new word and randomly adding that many letters to the word. 

If the listener has the utterance in its vocabulary, the conversation is deemed as a success, and the weight of the edge between the agents in the social network is increased, the personalities of both the listener and speaker are altered, and the weight of the word in the vocabularies of both agents are increased. If the listener did not have the word in its vocabulary, the conversation is deemed a failure, and the weight of the edge between the agents in the social network decreases. The speaking agent decreases the weight of the word in its vocabulary, while the listener adds the word into its vocabulary. None of the agents' personalities are altered when the conversation is unsuccessful.

%%=========================================