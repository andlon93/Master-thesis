\acresetall
%%=========================================
\chapter{Summary and Recommendations for Further Work}
This section will summarise the previous sections and present a suggestion for future work based on the findings.

\section{Summary and Conclusions}
% Here, you present a brief summary of your work and list the main results you have got. You should give comments to each of the objectives in Chapter 1 and state whether or not you have met the objective. If you have not met the objective, you should explain why (e.g., data not available, too difficult). This section is similar to the Summary and Conclusions in the beginning of your report, but more detailed—referring to the the various sections in the report.
In language evolution, there are three theories: Language evolution can be explained by biological evolution \citet{chomsky1986knowledge}, by natural selection \citet{pinker1990natural}, or by society and culture affecting language \citet{quillinan2006social}. These three theories can be associated with one of the three forces of language evolution: biological evolution, cultural evolution, and learning. It is likely that all three of these factors affect how language evolves in some manner, but how is still unclear. 

The Baldwin effect was first proposed by Mark Baldwin in 1896 \citet{baldwin1896new}. The theory states that learning during an individuals lifetime can help guide the evolution. If the selection pressure is high enough, the individuals that learn a certain skill easily will have a greater chance of surviving. If the environment stays the same, natural selection will be guided to the individuals that easily learn the skill. There has been made many models testing this theory \citet{lipowska2011naming, zollman2010plasticity, chater2010language}, and most models seem to indicate that the Baldwin effect has an effect on language evolution. To what degree it affects language evolution is still uncertain.

When language evolution is modeled, it is done through language games. A language game involves two agents chosen from the population. These two agents then conduct a conversation. Then, based on how the conversation went, they update their vocabularies. How language is modeled varies a lot, it can be holistic \cite{lipowska2011naming}, signals sent from an \ac{ann} \cite{munroe2002learning}, or it can be compositional \cite{gong2004computational}. 

Some models have included social networks which tries to simulate how people group together. The most common method for modeling a social network is using a weighted undirected discrete graph. The relationship between the vertices, or agents, have been modeled solely based upon their ability to communicate with each other. These models choose the agents to converse based on the connections in the social network. Some models have the parents conduct the first conversations with their children before they can converse with other agents \citet{lipowska2011naming, munroe2002learning}. The fitness functions of these models vary from directly using an agent's communicative success as an agent's fitness \citet{lekvam2014co, lipowska2011naming}, to having the agents use the input to perform actions in the real world and measuring the fitness based on how well the action is performed \cite{munroe2002learning}. 

None of the models seem to factor in personality in their models. The Schwartz Theory of Basic Values made by Shalom H. Schwartz tries to represent personalities based on which values one have. The principle is that a person’s values  represent which goals one have. Assuming that a person is trying to achieve those goals, one can deduce which actions are most likely to fulfil those goals. When knowing a person’s goals, one knows that person’s personality. This research also found that there was a consensus across nations which three values was the most important and which three was least important.

\section{Discussion and future work}
Language games is utilised in almost all computational models concerning language evolution. Language games are massive simplifications of human language, but they capture the main idea of language evolution. The idea is that a population starts with a need to communicate, but has no way to do it. From this, the agents attempt to communicate and eventually reach a common language. 

Many recent language games incorporate social networks into their language game models. Social networks affect who people choose to converse with, which affects an agent’s language because they learn through conversations. Social networks are important if one wants to understand the effects cultural learning has on language because social networks affect an individual’s language. The social network used in most models are simple. All agents are assumed to be extrovert, looking for new connections. It is also assumed that if two agents with the same language have a conversation, the conversation will always be redeemed as a success and the agents will develop a strong connection. This is obviously not the case in reality, many people with the same language can not connect because of their personal differences. If a person does not fit into a group, that person might look for a group that fits the person better, or slowly personality in order to fit in with the group. 

Using the theory of Basic Values presented in section \label{SchwartzBasicValues}, the agents will for example have values stating whether they care more about themselves than the better of the group, whether they are extrovert or introvert, or whether they are leaders or followers. Creating a model with the agents having a personality will make language evolution on both the ontogentic and the glossogenetic time scale more realistic. 

An individual's fitness should display how well that person fits into society, e.g. how well his language is and whether his values fits the values of a group. Most fitness functions have been solely based upon the language of each individual. The fitness of an agent should represent how well that person fits into a group, meaning it is a combination of how well his personality fits the group and how well he can communicate with the others in the group. 

Incorporating this into an already existing language game based model with a social network seems natural since a language game based model probably is suited for the inclusion of personalities. Lipowskas model uses a lattice and not a social network, but that is the only part that would have to changed. The model already uses a naming game, which is the best suited language game the suggested research. Munroe and Cangelosi's model does not investigate how social networks affect language evolution, and it utilises a signaling game rather than a naming game. This model would require so much work that making a new model from nothing would be easier. Gong's model utilises a spatial naming game and a complex compositional language in his model. For the purposes of researching the importance of social networks, utilising such a complex modeling of language seems unnecessary. The model made by Lekvam seems like the strongest candidate. This model already has a social network and a naming game. The model evolutionary part of the model utilises a \ac{ga} which seems like a good base for simulating language evolution. According to the paper \citep{lekvam2014co}, the model is well suited for extensions, and extending it with a more sophisticated social network may provide knowledge concerning how social evolution and social networks have affected the evolution of language. 
%%=========================================