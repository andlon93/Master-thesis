\acresetall
%%=========================================
\chapter{Discussion}
In the main experiment, the evolution was in favour of agents that acted extrovert during the first generation, because most dialogues were unsuccessful. As the number of unique words decreased, introvert dialogues became more likely to be successful, and a successful dialogue increased the involved agents' fitness. An extrovert agent had many edges, which increased its fitness. On the other hand, those edges would have a lower average weight than an agent with fewer edges and a low average weight would decrease the agent's fitness. So there was a trade-off between acting introvertly and having fewer edges with high weights, and acting extrovertly and having more edges with lower weights. The population evolved towards favouring introvert characteristics. Agents also evolved towards higher learning rate, most likely because a high learning rate made the language easier to learn. When the population had almost reached consensus, and the average vocabulary length was close to $1.0$, the average learning rate stopped increasing. This might have happened because the newborns were only taught one word which made the learning unimportant. 
 
The first ten generations, the number of unique highest ranked words dropped. The agents that got edges of a certain weight survived the first generations. Those agents had a high fitness and were probable to be chosen as parents several times during the parent selection phase. The agents that got selected as parents several times and had a high probability of speaking to their children acquired more edges with strong weights, which increased their fitness and likelihood of surviving to the next generation. This cycle kept a select few agents alive for the first generations, allowing them to spread their language to many others. This is how the number of unique words dropped quickly during the first generations of the simulation. 

The average vocabulary length increased during the first seven generations because most did not have one word with a much higher weight than the others. Also, most children were taught at least two words, usually more, during the first generations. Most of the words they were taught got a very low weight and less words were passed on as the generations passed by. In order to gain edges during the first generation, the agents acted extroverted, which lead to them learning or teaching new words which made the average vocabulary size increase. When a few words in the vocabularies had gotten a significantly higher weight than the others, those words were the only words that the newborn agents learned. The process where newborns only picked up the most successful words of their parents quickly reduced the number of unique words and the average vocabulary length. After the first ten generations, the evolution towards consensus continued slowly. 
 
The social network became denser as the population evolved, up until generation $40$ which was when the average degree of the agents stabilised. Even though the agents mainly acted introvertly, the social network was not divided into groups, like in Lekvam’s experiment \citep[Section 5.1]{lekvam2014co}. The reason why this is the case probably was because every agent could contact any other agent when it acted extrovertly. This meant that the entire population existed without any physical distance from one another. A community of individuals living so close to each other might evolve a social network similar to the one from this simulation. The network consisted of several circular layers, where the innermost layer had agents with the most edges, and the outer layer had agents with the fewest edges.

Experiment 2 reached consensus after fewer generations than experiment 1. This happened because there were fewer words to begin with and fewer agents to teach one word to. One agent with a high probability was more likely to become a parent in a small population, which lead to more newborns that learned the same word from an earlier generation. The tendencies concerning how the language evolved were similar, only that it all happened faster with higher variance. Especially the average degree and fitness had big variations from one generation to another. This is natural considering that there were randomness involved in the form of choosing the agents involved in dialogues, and making children. When fewer dialogues and fewer children were made per generation, the variance increase.

Experiment 3 should have less variance and have required more generations to reach consensus than experiment 1 if this reasoning was correct, and it had less variance and reached consensus later. The evolution began similarly the two previous experiments, where the population's average degree, fitness and unique words evolved quickly. The population stabilised on a high number of unique words and a fitness of $0.5$. At generation $60$, the parents taught their children a few words and the average vocabulary length started falling, and the average fitness increased. This is exactly what happened in experiment 1, only that the population in experiment 3 required more time to reach a state where the words had a significantly higher weight, because of the population’s size. 

The social network had small groups of two agents for many generations. Those small groups eventually died out or joined the main group because the small groups were unable to get a big enough fitness on their own. This happened more slowly because the newborns in the main group learned many different words from their parents, giving the agents in the main group a lower average fitness than the main group in the two previous experiments.

The number of dialogues per generation was quintupled in experiment 4, and it would be natural to assume that it would allow the population to reach consensus faster than in the main experiment, but it did not. It did, however, provide the agents with a higher average fitness because the average weight increased a lot due to the high number of introvert dialogues. The average degree decreased after the second generation because the parents taught their newborn agents fewer words than what they had in their own vocabularies. However, the average quickly increased again, because the newborns shared their learned words through extrovert conversations. The number of extrovert conversations per agent per generation was naturally higher in this simulation. That lead to more newborn agents dying since their parents had a very high fitness. The parents' fitness was high because they had conducted so many dialogues during their lifetime that they had several highly weighted connections. This simulation required the same number of generations to reach consensus as experiment 1, which indicates that the number of dialogues had little effect on reaching consensus. This result lead to the conclusion that the dialogues had a small impact on the agents' vocabulary. The only other method for a newborn to learn a language in this model was through initial conversations with its parents, and this experiment indicated that parent-child dialogues might be important for this model to reach consensus.  

The next experiment tested the hypothesis that the conversations a newborn has with its parents were of great importance in order for the population to reach a consensus. This was tested through the probability of parents speaking to their newborn being set to three times less than in the main experiment. If the hypothesis was to be true, this experiment should have required more generations than the main experiment to reach consensus. Throughout the experiments performed thus far, when the average vocabulary length started decreasing, it seemed to be an indicator of that the population was approaching consensus. In the main experiment, the average vocabulary size started decreasing at generation eight, but in this experiment, the average vocabulary size started decreasing at generation $25$. The main experiment had reached consensus at generation $61$, while this experiment reached consensus at generation $127$. This result supports the hypothesis that this computational model's most important factor for reaching consensus is the parents speaking to their children. This also indicates that the first dialogues are essential for the newborns language in this computational model. The language of the children in this computational model is defined through its first dialogues, and after that it is very hard to change primary language, just like humans who tends to struggle when attempting to learn a language at an adult age. 

The average degree was significantly lower than in the main experiment because a lot of newborns did not get the initial dialogues with their parents. The parent-child dialogues were reduced but the experiment had the same number of dialogues, and therefore the same number of extrovert dialogues. This indicates that the average degree should be roughly $1/3$ of the main experiment. The main experiment stabilised on an average degree of about $10$, and this experiment stabilised on an average of about $4$, which is $2.5$ times as few edges as the main experiment.

The social network in this experiment evolved through a few agents, which had a much higher fitness and a much higher probability of speaking to their children than the other agents in the population had. Those parents created the fan-like shapes, which was visible at generation $5$ and $10$. At generation $100$, the social network had evolved into a network similar to the network at generation $10$ from the main experiment. The network did not change during the last $50$ generations of the simulation. The fact that the social network in this simulation was much less dense indicate that the connections gotten from the initial dialogues were important when it comes to acquiring other connections than an agents parents. This probably is because the probability of becoming a speaker is related to an agents' fitness. An Agent without connections have a fitness of zero, making it unlikely that it would become a speaker. In this experiment, many newborns did not have dialogues with their parents, which lead them to be unlikely to be chosen as speakers, which again lead to the social network being less dense compared to the main experiment. All features that were affected by the language in this experiment evolved similarly to the main experiment, only significantly slower, which indicates that parents speaking to their children had a big effect on the evolution towards consensus.

If the number of newborns being made per generation were less, one could expect to see similar results as the previous experiment because, with fewer newborns per generation, fewer newborns would conduct initial dialogues with their parents. Experiment 6 tested this through increasing the $k$ and $n$ value. The results show that the population struggled to reach consensus, in fact, the population did not reach consensus within the $150$ generations the simulation was performed. The result indicates that it is not only important to have many newborns being taught a consistent language in the early years, it is also important that a certain number of newborns are made each generation to allow the evolution to move forward. 

When a larger portion of the population survived each generation, the selection pressure was lessened. When the selection pressure was lessened, the evolutionary forces at the \textit{phylogenetic} and \textit{glossogenetic} had less of an impact. See section \ref{EvoltionaryForces} for a description of the evolutionary forces. The second experiment performed with varying turnover had an even smaller selection pressure, which lead to the language evolving even slower. These results indicate that when the selection pressure on a species is small enough it does not need to evolve in order to survive, and then the species will evolve very slowly. In this computational model, when the selection pressure was high enough, the population reached consensus, but when the selection pressure was lessened the evolution towards consensus slowed down. This indicates that the Baldwin effect affects the \textit{phylogenetic} and \textit{glossogenetic} forces of language evolution. 

The average fitness and degree was higher when a larger part of the population survives to the next generation. The agents who survive generally are among the fittest of the current population, and so when a larger portion of the agents with a high fitness survive, the average fitness becomes higher. The average degree is directly related to the fitness function, so it is natural that it is higher as well. 

When the agents could make up new words with a probability of $0.25$, one would expect the percentage of successful dialogues to be less and average vocabulary size to be more than in the main experiment. One might also expect that the population would have required more generations to reach consensus when a lot more utterances existed. The average fitness, degree, and percentage of successful dialogues were lower than in the main experiment, and the average vocabulary size was a lot higher. This did, however, not significantly affect the evolution towards consensus. 

The previous simulations that had struggled to reach consensus had forced the population to conduct less parent-child dialogues. This simulation did not force less parent-child dialogues. In fact, the probability of a parent choosing to utter a word from its vocabulary two out of the three dialogues it has with its newborn is $0.84$. The probability of at least one parent uttering at least two words from its vocabulary is $1 - (1-0.84)^2 = 0.97$. So the parent-child dialogues were not as affected by this as one might expect them to be. Also, the added number of words had very little impact, because they were weighted very low compared to the words that had survived for several generations. 

When the probability of inventing a word was increased to $0.4$, the simulation struggled to reach consensus. It seems that with a probability that high, several of the newborns were taught several words from their parents leaving the newborns without consistent language to teach the next generation. When that happened to a large enough portion of the population, new words actually appeared in smaller groups for a few generations until the branch keeping the word alive died. The appearing and disappearing of these smaller languages happened regularly, but most of the population still had agreed upon one word after about $60$ generations. This did not affect the population's ability to agree upon one word because almost the entire population was in consensus. The simulation did, however, allow new words to appear from time to time even though one word was very dominant. 